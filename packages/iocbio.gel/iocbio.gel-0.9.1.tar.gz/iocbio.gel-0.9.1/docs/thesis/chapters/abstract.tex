As part of this thesis, a desktop application for Tallinn University of Technology's Cybernetics Institute's Systems Biology laboratory was made. The goal was to optimize the current protein analysis workflow, speeding up certain tasks and operations that required unnecessary repetition and precision.

The software was built on the Qt framework. It was written in Python, using the PySide6 wrapper library. It uses either SQLite or PostgreSQL for data storage and the filesystem or OMERO for images. 

The result is a software that can load gel images and crop, rotate, perform background subtraction on them. Lane regions can be defined on the image for intensity plot generation, which can be cut off at places to extract desired areas. Those results can be stored in a database for later review and analysis.

The software was planned to be released to the public as an open-source project. Towards the end of development, however, it was considered not to be ready yet, requiring more polish.

The project's GitLab repository can be found at \url{https://gitlab.cs.ttu.ee/jakutt/iaib}.

The thesis is in English and contains 30 pages of text, 8 chapters, 25 figures, 9 tables.
