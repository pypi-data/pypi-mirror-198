% Overwrite local hyphenation formatting to avoid contradicting est language rules
{
  \hyphenpenalty=100000
  \emergencystretch=\maxdimen

Bakalaurusetöö raames loodi Tallinna Tehnikaülikooli Küberneetikaosakonna Süsteemibioloogia laborile töölauarakendus. Rakenduse eesmärk on senise geelianalüüsi tööprotsessi optimiseerimine, kiirendades teatud toiminguid mis nõudsid ebavajalikul määral korduvaid liigutusi ja täpsust.

Tarkvara ehitati Qt raamistikule. Rakendus loodi Pythonis, kasutades PySide6 mähisteeki.
Andmetalletuseks on valida SQLite ja PostgreSQLi andmebaaside vahel, ning piltide laadimiseks failisüsteemi või OMERO andmebaasi.

Tulemuseks on tarkvaralahendus, mis suudab geelipilte sisse laadida, pildilt alamosa valida ning seda pöörata, ning teostada taustaeemaldust. Pildile saab paigutada radu, mille põhjal koostatakse intensiivsusgraafe, mida saab vajadusepõhiselt piiritleda nõutud pindala kättesaamiseks. Tulemusena saadud andmeid saab talletada andmebaasi hilisemaks ülevaatluseks ja analüüsiks.

Alguses planeeriti valminud rakendus avaldada publikule avatud lähtekoodiga, aga lõpupoole selgus, et rakendus pole veel piisavalt valmis, nõudes lisaviimistlusi.

Projekti GitLabi salv on leitav aadressilt \url{https://gitlab.cs.ttu.ee/jakutt/iaib}.

Lõputöö on kirjutatud inglise keeles ning sisaldab teksti 30 leheküljel, 8 peatükki, 25 joonist, 9 tabelit.

}
